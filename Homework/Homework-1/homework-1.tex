
\documentclass[11pt]{article}

\usepackage[margin=1in]{geometry}
\usepackage{amssymb,amsmath}
\usepackage[final]{graphicx}
\graphicspath{ {./images/} }

\pagestyle{fancy}
\fancyhead[L]{\textsc{Chapter 2 Homework: 27, 30, 43, 49, 50, 55, 57}}
\fancyhead[C, R]{}
\fancyfoot[L]{}
    \fancyfoot[R]{\thepage}
\fancyfoot[C]{}

\usepackage[hyphens]{url}
\usepackage{hyperref}
\hypersetup{
    colorlinks=true,
    linkcolor=blue,
    filecolor=magenta,      
    urlcolor=cyan,
    citecolor = black
} 

\usepackage{titlesec}
\titleformat{\section}[block]{\Large\bfseries\filcenter}{}{1em}{}
\titleformat{\subsection}[hang]{\bfseries}{}{1em}{}
\setcounter{secnumdepth}{0}

\begin{document} 

\begin{enumerate}
\item[27.]{If children obtain half their genes from one parent and half from the other parent, why aren’t siblings identical?}

\item[]{Because the “half” inherited is very random, the chances of receiving exactly the same half is vanishingly small. Ignoring recombination and focusing just on which chromosomes are inherited from one parent, there are 223 = 8,388,608 possible combinations!}

\item[30.]{Four of the following events are part of both meiosis and mitosis, but only one is meiotic. Which one? (1) chromatid formation, (2) spindle formation, (3) chromosome condensation, (4) chromo- some movement to poles, (5) synapsis} 

\item[]{(5) chromosome pairing (synapsis)}

\item[43.] {a. The ability to taste the chemical phenylthiocarbamide is an autosomal dominant phenotype, and the inability to taste it is recessive. If a taster woman with a nontaster father marries a taster man who had a nontaster daughter in a previous marriage, what is the probability that their first child will be:\\
(1) a nontaster girl?\\
(2) a taster girl?\\
(3) a taster boy?\\
b. What is the probability that their first two children will be tasters of either sex?}
\end{enumerate}


\end{document}
